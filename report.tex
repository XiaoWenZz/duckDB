\documentclass[UTF8]{ctexart}
\usepackage{graphicx}
\usepackage{booktabs}
\usepackage{hyperref}
\usepackage{listings}
\usepackage{xcolor}
\usepackage{geometry}
\usepackage{float}

\geometry{left=2.5cm,right=2.5cm,top=2.5cm,bottom=2.5cm}

\title{\textbf{DuckDB 并行分析性能评测报告}}
\author{性能验证小组}
\date{\today}

\begin{document}

\maketitle

\section{引言 (Introduction)}
本报告详细记录了对 DuckDB 并行分析处理能力的验证与性能评测结果。本次评测的主要目标是分析系统在不同线程数下的扩展性(Scalability),以及对比不同数据格式(Parquet 与 CSV)对分析性能的影响。

\section{实验环境与设置 (Experimental Setup)}

\subsection{硬件环境}
\begin{itemize}
    \item \textbf{CPU}: Intel(R) Core(TM) i9-14900K (24 Cores, 32 Threads)
    \item \textbf{内存}: 128GB
    \item \textbf{操作系统}: Ubuntu 22.04.4 LTS
    \item \textbf{存储}: 2TB NVMe SSD (KINGSTON SNV3S2000G)
\end{itemize}

\subsection{数据集}
\begin{itemize}
    \item \textbf{来源}: NYC Taxi Data (Yellow Trip Data, 2019-2021).
    \item \textbf{格式}: Parquet (列式存储) 与 CSV (行式文本存储).
    \item \textbf{数据量}: 约 30GB+.
\end{itemize}

\section{实验设计 (Experimental Design)}

\subsection{查询负载 (Workloads)}
本实验选取了三个具有代表性的 OLAP 查询,分别涵盖了 I/O 密集型、混合型和 CPU 密集型场景。这些查询基于 NYC Taxi 数据集的真实分析需求设计。

\subsubsection{Q1: 简单聚合 (I/O Bound)}
该查询计算整个数据集的总行数和平均总金额。由于涉及全表扫描且计算逻辑简单,其性能主要受限于 I/O 带宽。
\begin{lstlisting}[language=SQL, basicstyle=\ttfamily\small, frame=single]
SELECT count(*), avg(total_amount)
FROM 'data/yellow_tripdata_*.parquet';
\end{lstlisting}

\subsubsection{Q2: 过滤与分组 (Mixed)}
该查询包含过滤条件 (\texttt{WHERE}) 和分组聚合 (\texttt{GROUP BY})。DuckDB 的向量化执行引擎应能有效利用 SIMD 指令加速过滤过程。
\begin{lstlisting}[language=SQL, basicstyle=\ttfamily\small, frame=single]
SELECT passenger_count, avg(trip_distance) AS avg_dist
FROM 'data/yellow_tripdata_*.parquet'
WHERE trip_distance > 0 AND total_amount > 0
GROUP BY passenger_count
ORDER BY avg_dist DESC;
\end{lstlisting}

\subsubsection{Q3: 高基数分组与排序 (CPU Bound)}
该查询对两个高基数列 (\texttt{PULocationID}, \texttt{DOLocationID}) 进行分组并排序。这将产生大量的分组键,对哈希表构建、洗牌 (Shuffle) 和排序算法的并行效率提出较高要求。
\begin{lstlisting}[language=SQL, basicstyle=\ttfamily\small, frame=single]
SELECT PULocationID, DOLocationID, count(*) AS trip_count
FROM 'data/yellow_tripdata_*.parquet'
GROUP BY PULocationID, DOLocationID
ORDER BY trip_count DESC
LIMIT 10;
\end{lstlisting}

\subsection{变量控制}
\begin{itemize}
    \item \textbf{并行度 (Parallelism)}: 控制线程数 $N \in [1, 2, 4, 8, 16]$,验证 Amdahl 定律。
    \item \textbf{数据格式 (File Format)}: 对比 Parquet (列存) 与 CSV (行存),分析 I/O 与解析开销。
    \item \textbf{冷启动 vs 热启动}: 为模拟真实大规模数据分析场景,本次实验主要关注文件系统缓存产生的热启动性能,但也考虑了 I/O 对 Q1 的显著影响。
\end{itemize}

\section{结果与分析 (Results and Analysis)}

\subsection{实验一:并行度扩展性测试}
下表展示了在 Parquet 格式下,随着线程数增加,各查询的执行时间(秒)。

\begin{table}[H]
    \centering
    \caption{执行时间 vs. 线程数 (Parquet)}
    \label{tab:scalability}
    \vspace{0.2cm}
    \begin{tabular}{c|ccc}
        \toprule
        \textbf{线程数 (Threads)} & \textbf{Q1 (秒)} & \textbf{Q2 (秒)} & \textbf{Q3 (秒)} \\
        \midrule
        1 & 1.12 & 2.93 & 1.99 \\
        2 & 0.57 & 1.47 & 1.04 \\
        4 & 0.31 & 0.77 & 0.60 \\
        8 & 0.20 & 0.51 & 0.39 \\
        16 & 0.20 & 0.35 & 0.30 \\
        \bottomrule
    \end{tabular}
\end{table}

\textbf{分析总结}:
\begin{itemize}
    \item \textbf{扩展性趋势}: 随着线程数从 1 增加到 4,所有查询均表现出显著的性能提升,接近线性加速。这表明 DuckDB 的 Morsel-Driven Parallelism 架构在多核调度上非常高效。
    \item \textbf{收益递减}: 在 8 线程及 16 线程时,Q1(I/O 密集型)的耗时稳定在 0.20s 左右,不再随线程数增加而降低。这表明此时系统瓶颈已从 CPU 计算能力彻底转移至内存带宽或 I/O 读取速度。相比之下,计算更复杂的 Q2 和 Q3 在 16 线程时仍有微小提升。
    \item \textbf{资源利用率}: Q3 的高基数聚合涉及大量的哈希表操作,其良好的扩展性主要得益于 DuckDB 的并行哈希聚合算法(Radix Partitioning)。
\end{itemize}

\subsection{实验二:数据格式对比测试}
下表对比了在 16 线程下,处理 Parquet 格式与 CSV 格式的性能差异。

\begin{table}[H]
    \centering
    \caption{Parquet vs. CSV 性能对比 (16 线程)}
    \label{tab:formats}
    \vspace{0.2cm}
    \begin{tabular}{l|cc|c}
        \toprule
        \textbf{查询} & \textbf{Parquet (秒)} & \textbf{CSV (秒)} & \textbf{加速比 (CSV/Parquet)} \\
        \midrule
        Q1 (I/O Bound) & 0.20 & 4.78 & \textbf{23.9x} \\
        Q2 (Mixed)     & 0.35 & 4.70 & \textbf{13.4x} \\
        Q3 (CPU Bound) & 0.30 & 4.47 & \textbf{14.9x} \\
        \bottomrule
    \end{tabular}
\end{table}

\textbf{深入分析}:
\begin{enumerate}
    \item \textbf{列式读取}: Parquet 是列式存储格式,对于 Q1 这种只涉及两列 (\texttt{count(*)}, \texttt{total\_amount}) 的查询,DuckDB 仅需读取相关的数据块,显著减少了磁盘 I/O。而 CSV 必须扫描整行数据。
    \item \textbf{二进制 vs 文本}: Parquet 采用二进制编码(如 SNAPPY 压缩),无需进行繁重的文本解析(Text Parsing)。CSV 的解析(查找分隔符、转换字符串为数字)消耗了大量 CPU 周期,导致在 CPU 并不是瓶颈的 Q1 中也表现极差。
    \item \textbf{统计信息利用}: Parquet 文件头包含了 Min/Max 等统计信息。DuckDB 利用这些元数据在 Q2 中进行谓词下推(Predicate Pushdown),直接跳过不满足 \texttt{trip\_distance > 0} 的数据块,进一步拉大了性能差距。
\end{enumerate}

\section{结论 (Conclusion)}
实验表明,DuckDB 在多核环境下表现出优秀的并行扩展能力。同时,数据格式的选择对性能至关重要:在分析场景下,使用 Parquet 相比 CSV 可带来数量级的性能提升。建议在生产环境中优先将数据转换为 Parquet 格式以获得最佳性能。

\end{document}
